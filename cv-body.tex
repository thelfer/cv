\section{Domaine d'activité et projet professionnel}

Mes activités de recherche tournent essentiellement autour des
thématiques suivantes~:

\begin{itemize}
  \item Modélisation du comportement fragile et viscoplastique du
  combustible nucléaire oxyde.
  \item La capitalisation des lois de comportements mécaniques
  dans le cadre du groupe de travail tripartite \og~Validation à effets
  séparés~\fg{} qui cherche à mettre en place, avec les laboratoires
  expérimentaux du CEA (DEC,DMN) et d'EDF (MMC) une méthodologie et des
  outils communs garantissant une cohérence entre l'identification des
  lois de comportement mécanique et leur utilisation dans les codes de
  calcul. Dans ce cadre, je co-développe l'outil d'identification
  \texttt{LC0D}.
  \item Développement des capacités de simulation de la
  plateforme \texttt{PLEIADES} avec notamment des contributions majeures
  aux développements des projets \texttt{MANTA}, \texttt{MFEM-MGIS} et
  de l'outil de calcul scientifique \texttt{LICOS}.
  \item Implantation numérique des lois de comportement mécanique
  et leurs intégration dans les codes d'intérêt pour le CEA
  (\texttt{Manta}, \texttt{Cast3M}, \texttt{Europlexus},
  \texttt{AMITEX\_FFTP}) et ses partenaires industriels
  (\texttt{Cyrano}, \texttt{code\_aster}, \texttt{Cathare}).
  \item Modélisation thermo-mécanique des éléments combustible au
  sein de la plate-forme \texttt{PLEIADES}, ce qui couvre les aspects
  physiques (grandes transformations) et numériques (amélioration des
  performances, incompressibilité, robustesse):
  \begin{itemize}
    \item La simulation du comportement des combustibles REP en
    situation accidentelle de type APRP ou RIA dans l'application
    combustible \texttt{ALCYONE}~;
  \end{itemize}
\end{itemize}

Depuis la mise en open-source de \texttt{MFront} en 2014,
j'interagis avec un large spectre d'utilisateurs industriels (EDF,
Framatome, BGE Tech, PSA, Ariane Group, Safran Tech) et académiques
(CEA, CNRS, IRSN, etc.), français et étrangers (Allemagne, Espagne,
Itaglie, Royaume-Unis, États-Unis, Chine, etc.) sur différentes
thématiques d'intérêt pour les activités du département:

\begin{itemize}
  \item L'implantation de lois de comportement complexes, telles
  que les lois de plasticité cristalline. D'abord développées au DMN,
  ces lois sont devenues centrales dans les travaux de recherche
  micro-mécanique du SESC (thèses de Julian-Soulacroix, Luc Portelette,
  Julien Labat, Hakima Bouizem).
  \item le développement des solveurs multiphysiques qui
  répondent aux besoins de flexibilité des ingénieurs de recherche et
  aux besoins HPC de la plateforme. Les interactions avec 
\end{itemize}

\section{Points remarquables}

\begin{itemize}
\item Membre du groupe de travail CEA/EDF/Areva) sur \og{}~la
  capitalisation des lois de comportement en support à la
  qualification unitaire des codes~\fg{}.
\item Correspondant technique sur la modélisation mécanique dans la
  plate-forme PLEIADES~;
\item Correspondant technique sur les lois de comportement gaine pour
  le projet CRAYON~;
\item Enseignant dans le Materials science for Nuclear Energy
  (MaNuEn)/European Master in Nuclear Energy (EMINE)
\item Enseignant à Polytech’Marseille, option \og{}~Structures et
  Ouvrage\fg{}, 3ème année cursus ingénieur
\item Reviewer pour de nombreux journaux.
\end{itemize}

\section{Projets open-source}

\begin{itemize}
  \item Le générateur de code \texttt{MFront} et le projet \texttt{TFEL}:
  \begin{itemize}
    \item \url{https://thelfer.github.io/tfel/web/index.html}
    \item \url{https://github.com/thelfer/tfel}
    \item \url{https://www.researchgate.net/project/TFEL-MFront}
    \item \url{https://twitter.com/TFEL_MFront}
  \end{itemize}
  \item Le projet \texttt{MGIS} (\texttt{MFrontGenericInterfaceSupport}):
  \begin{itemize}
    \item \url{https://thelfer.github.io/mgis/web/index.html}
    \item \url{https://github.com/thelfer/MFrontGenericInterfaceSupport}
  \end{itemize}
  \item Le projet \texttt{MFEM/MGIS}:
  \begin{itemize}
    \item \url{https://thelfer.github.io/mfem-mgis/web/index.html}
    \item \url{https://github.com/thelfer/mfem-mgis}
  \end{itemize}
  \item Le projet \texttt{mgis.fenics}:
  \begin{itemize}
    \item \url{https://thelfer.github.io/mgis/web/mgis_fenics.html}
  \end{itemize}
  \item Le projet \texttt{MFrontGallery}:
  \begin{itemize}
    \item \url{https://github.com/thelfer/MFrontGallery}
  \end{itemize}
\end{itemize}

\section{Parcours professionnel}

\begin{tabular}[htbp]{p{0.1\linewidth}p{0.9\linewidth}}
  {\tiny 2015-2022} & Correspondant technique pour l’évolution des composants mécaniques de PLEIADES \\
  {\tiny 2015-2022} & Correspondant technique pour les lois de comportement gaine du projet CRAYON \\
  {\tiny 2013-2022} & Responsable du développement de MFront \\
  {\tiny 2015-2021} & Modélisation des accidents graves REP (RIA/APRP) \\
  {\tiny 2013-2015} & DEC/SESC/LSC: Développement GERMINAL (éléments absorbants, composants RAMSES et thermohydraulique) \\
  {\tiny 2009-2014} & DEC/SESC/LSC: Responsable de développement de Licos : applications pour la conception des éléments combustibles et absorbants innovants, des irradiations non standards et à l’interprétation des irradations non standards \\
  {\tiny 2010-2008} & DEC/SESC/LSC: Responsable de dévéloppement du code de calcul CELAENO pour les éléménts combustible de la filière RNR-Gaz et le design des irradiations expérimentales associées \\
\end{tabular}

\section{Formation universitaire}


\cventry{2005}{Docteur en mécaniques des solides}{École centrale de
  Lyon}{France}{Étude de l'impact de la fissuration des combustibles
  nucléaires oxyde sur le comportement normal et incidentel des
  crayons combustible}{}

\cventry{2002}{DEA en Sciences des Matériaux}{Université Henri
  Poincaré (Nancy I)}{France}{}{}

\cventry{2002}{Ingénieur}{École des Mines de Nancy}{France}{Spécialité
  en science et ingénierie des matériaux}{}

\section{Encadrement}

\cventry{2020-2022}{Doctorat de David
  Siedel}{Méthode hybride high-order appliquée à la description de la
  fissuration des combustibles nucléaires par champs de phase.}{Doctorat
  co-encadré par O. Fandeur (DM2S).}{}{}

\cventry{2019-2021}{Doctorat de
  Jean-Baptiste Parise}{Études des mécanismes de déformation
  viscoplastique du dioxyde d'uranium polycristallin au voisinage de la
  stoechiométrie - influence de l'activité d'oxygène et de la
  microstructure}{Thèse dirigé par Philippe Garcia.}{}{}

\cventry{2019}{Post-doctorat de Ye
  Lu}{Description de la fissuration fragile par champs de phase:
  application aux combustibles nucléaires par champs de phase et aux
  bétons}{Post-doctorat co-encadré avec Olivier Fandeur (DM2S) et Benoît
  Bary (DPC)}{}{}