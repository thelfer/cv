\documentclass[11pt,a4paper]{moderncv}

\usepackage[a4paper,margin=3cm]{geometry}
\usepackage[utf8]{inputenc}
\usepackage[T1]{fontenc}
\usepackage{multibib}
\usepackage[frenchb]{babel}
\usepackage{etoolbox}
\BeforeBeginEnvironment{thebibliography}{%
  \let\origsubsection\subsection% save the original definition of \section
  \let\subsection\section%  make \section behave like \subsection
}
\AfterEndEnvironment{thebibliography}{%
  \let\subsection\origsubsection% restore the original definition of \section
}

\newcommand{\paragraph}[1]{\textbf{#1}}

\makeatletter
\newcommand\@pnumwidth{1.55em}
\newcommand\@tocrmarg{2.55em}
\newcommand\@dotsep{4.5}
\makeatother
\usepackage{titletoc}

\makeatletter
\setlength\columnsep{20pt}
\setcounter{tocdepth}{1}
\newcommand\contentsname{Contents}
\newcommand\tableofcontents{%
\addtocontents{toc}{\protect\setcounter{tocdepth}{0}}
\section*{\contentsname}
\addtocontents{toc}{\protect\setcounter{tocdepth}{2}}
    \@starttoc{toc}%
    }
\usepackage{titletoc}
\titlecontents*{section}[0pt]
  {}{}{}
  {, \itshape\thecontentspage}[\ \textbullet\ ][.]
\renewcommand\l@subsection[2]{}
\makeatother
\renewcommand{\contentsname}{Principaux thèmes}

% ma préférence en terme de design
\moderncvtheme[orange]{casual}
% d'autres valeurs possibles
\moderncvstyle{casual}
% Largeur de la colonne pour les dates
\setlength{\hintscolumnwidth}{1cm}
% Une entête classique
\firstname{Thomas}
\familyname{Helfer}

\title{ Modélisation et simulation du comportement thermo-mécanique
  des éléments combustibles\\ Laboratoire de simulation du combustible\\
  (CEA DEN/DEC/SESC/LSC) }

\address{Centre de Cadarache}{F-13108 Saint-Paul Lez Durance, France}
\email{thomas.helfer@cea.fr} \extrainfo{38 ans}
% d'autres valeurs possibles
\phone{33+ (0)4 42 25 22 67}
\fax{33+ (0)4 42 25 47 47}
% \extrainfo{Docteur }
%\mobile{06 00 00 00 00}
\homepage{www.tfel.sourceforge.net}
\photo[64pt][0.4pt]{img/moi.png}
%\quote{une citation}

\newcites{articles,proceedings,notes,talks,mfront}%
         {Articles,%
          Actes de congrès,%
          Notes techniques,%
          Communications orales,%
          Publications citant MFront}

\begin{document}

\maketitle

\setcounter{tocdepth}{0}
\tableofcontents{}

\section{Domaine d'activité et projet professionnel}

Je m'intéresse essentiellement~:
\begin{itemize}
\item à la modélisation thermo-mécanique des éléments combustible au
  sein de la plate-forme PLEIADES, ce qui couvre les aspects physiques
  (fissuration des matériaux fragiles, grandes transformations,
  incompressibilité) et numériques (amélioration des performances,
  robustesse)~;
\item à la simulation du comportement des combustibles REP en
  situation accidentelle de type APRP ou RIA~;
\item à la capitalisation des lois de comportements mécaniques pour
  lesquels je travaille à mettre en place, avec les laboratoires
  expérimentaux du CEA (DEC,DMN) et d'EDF (MMC) une méthodologie et
  des outils communs garantissant une cohérence entre l'identification
  des lois de comportement mécanique et leur utilisation dans les
  codes de calcul.
\end{itemize}

J'ai été le principal développeur de \texttt{LICOS}, une application
combustible dédiée aux éléments combustible et absorbant innovants
ainsi qu'aux irradiations expérimentales non standard. Grâce à cette
expérience, je me suis intéressé aux problématique de la plupart des
filières nucléaires d'intérêt pour le CEA (REP, RNR, RNR-G). La
qualité de \texttt{LICOS} a été saluée en interne et par les
partenaires industriels du CEA (AREVA/EDF).

En parallèle de \texttt{LICOS}, j'ai développé \texttt{MFront}, un
outil d'intégration de connaissances matériau, qui est au coeur de la
stratégie de mutualisation et de capitalisation de la plate-forme
PLEIADES. Depuis sa mise en open-source en 2014, cet outil connaît un
succès croissant bien au delà du domaine combustible, tant au CEA
(DMN, DER, DM2S, DPC) que chez ses partenaires industriels (EDF et
AREVA), mais aussi dans des cercles plus large que le nucléaire (CNRS,
Airbus, Communauté européenne, etc...). \texttt{MFront} est décrit
plus en détails ci-après.

\paragraph{Points saillants}

\begin{itemize}
\item membre du groupe de travail CEA/EDF/Areva) sur \og{}~la
  capitalisation des lois de comportement en support à la
  qualification unitaire des codes~\fg{}.
\item correspondant technique sur la modélisation mécanique dans la
  plate-forme PLEIADES~;
\item correspondant technique sur les lois de comportement gaine pour
  le projet CRAYON~;
\item 9 papiers publiés dans des journaux à comité de lecture~:
  \begin{itemize}
  \item dont 4 en tant que premier auteur
  \item 2 papiers supplémentaires sont en cours de finition
  \end{itemize}
\item 51 notes techniques:
  \begin{itemize}
  \item dont 34 en tant que premier auteur
  \end{itemize}
\item 10 participations à des actes de congrès
\item contribution directe à 7 faits marquants DEN/DEC
\item reviewer pour Nuclear Engineering and Design.
\end{itemize}

\subsection{Modélisation et simulation des éléments combustibles et absorbants}

\paragraph{La plate-forme \texttt{PLEIADES}}

\begin{center}
  \includegraphics[width=0.5\linewidth]{img/logo_pleiades2013.png}
\end{center}

\citetalks{helfer_recent_2015}
\citeproceedings{petry_advanced_2015}
\citeproceedings{lainet_recent_2013}
\citetalks{helfer11:_utilis_cast3_pleiad}
\citetalks{helfer10:_platef_pleiad}
\citeproceedings{plancq_pleiades_2004}

\paragraph{Simulation du comportement des éléments combustibles en
  situation normale et incidentelle}

\begin{center}
  \includegraphics[width=0.4\linewidth]{img/LeSaux-071.jpg}\quad
  \includegraphics[width=0.4\linewidth]{img/p.pdf}
\end{center}

\citearticles{guenot-delahaie_simulation_2018}

\citeproceedings{blanc_characterization_2015}
\citeproceedings{guenot-delahaie13:_state_astrid}
\citeproceedings{lainet_recent_2013}

\citearticles{sercombe_2d_2016}
\citeproceedings{sercombe_2d_2015}
\citearticles{sercombe_stress_2013}

\paragraph{Lois de comportements mécaniques~: le générateur de
  code \texttt{MFront}}

\begin{center}
  \includegraphics[width=0.5\linewidth]{img/mfront-en.png}
\end{center}

\texttt{MFront} couvre la plupart des applications en calcul de
structure (quasi-statique, dynamique, petites et grandes
transformations) et permet de décrire une grande variété de
comportements non linéaires (plasticité, viscoplasticité,
endommagement, etc..). Cette versabilité fait qu'on recense des
applications au combustible, aux métaux, aux bétons, aux sols, aux
polymères, etc... \texttt{MFront} garantit des performances numériques
excellentes, supérieures à celles des lois natives des codes éléments
finis usuels (\texttt{Cast3M}, \texttt{Code\_Aster},
\texttt{Abaqus/Standard}), une simplicité d'utilisation inégalée, et
la portabilité entre les codes des lois de comportement générées.

\texttt{MFront} est aujourd'hui un élément clé de la stratégie de
simuation du CEA et de ses partenaires. Ce produit est distribué dans
les produits suivants:
\begin{itemize}
\item la plate-forme \texttt{PLEIADES}~;
\item le code aux éléments finis \texttt{Code\_Aster}. Les lois
  générées par \texttt{MFront} ont remplacées la plupart des lois
  natives, moins performantes~;
\item la plate-forme \texttt{Salomé-Méca}~;
\item la plate-forme \texttt{MAP} (\texttt{Material Ageing
    PlateForm})~;
\end{itemize}
Il peut également être téléchargé à l'adresse suivante~:
\begin{center}
  \url{http://www.tfel.sourceforge.net}
\end{center}

\citearticles{portelette_crystal_2018}
\citearticles{guenot-delahaie_simulation_2018}
\citearticles{bary_analytical_2017}
\citeproceedings{helfer_phase_field_2017}
\citeproceedings{helfer_mfront_2017}
\citearticles{bary_thermoviscoelastic_2016}
\citearticles{helfer_introducing_2015}
\citeproceedings{petry_advanced_2015}
\citeproceedings{helfer_implantation_2015}
\citetalks{ramiere_acceleration_2017}
\citetalks{helfer_material_2017}
\citetalks{helfer_modelisation_2016}
\citetalks{helfer16:_presen_mfron}
\citetalks{helfer15:_presen_mfron-2}
\citetalks{helfer15:_presen_mfron}
\citetalks{hure14:_quelq_mfron}
\citetalks{helfer14:_presen_mfron}
\citetalks{helfer09:_mfron_cast3}

\paragraph{L'application de conception \texttt{LICOS}}

\begin{center}
  \includegraphics[width=0.45\linewidth]{img/assemblage-sfr.png}
  \hspace{2cm}
  \includegraphics[width=0.25\linewidth]{img/TH.png}
\end{center}

\begin{center}
  \includegraphics[height=0.75\linewidth,angle=-90]{img/Diamino9.png}
\end{center}

\citearticles{bejaoui17:_therm_diamin_licos}
\citearticles{helfer_licos_2015}
\citetalks{helfer13:_le_licos}
\citeproceedings{zabiegone:_inser_rbc_astrid}
\citeproceedings{bejaoui_thermomechanical_2013}
\citeproceedings{helfer_fuel_2009}

\subsection{Mécanique non linéaire des matériau}

\citearticles{portelette_crystal_2018}
\citearticles{bary_analytical_2017}
\citeproceedings{helfer_phase_field_2017}
\citeproceedings{helfer_mfront_2017}
\citearticles{michel16}
\citetalks{helfer_modelisation_2016}
\citearticles{helfer_extension_2015}
\citetalks{helfer15:_curren_pleiad}
\citeproceedings{helfer_two-dimensional_2006}

\paragraph{Comportement des matériaux combustible, des matériaux
  absorbant et des matériaux de gainage}

\citearticles{portelette_crystal_2018}
\citeproceedings{michel_3d_2016}
\citearticles{salvo_experimental_2015}
\citearticles{salvo_experimental_2015-1}

\subsection{Numérique}

\citearticles{michel_new_2017}
\citeproceedings{castelier_using_2016}
\citearticles{ramiere_iterative_2015-1}

\section{Parcours professionnel}

\begin{tabular}[htbp]{p{0.1\linewidth}p{0.9\linewidth}}
  {\tiny 2015-????} & Modélisation des accidents graves REP (RIA/APRP) \\
  {\tiny 2015-????} & Correspondant technique pour l’évolution des composants mécaniques de PLEIADES \\
  {\tiny 2015-????} & Correspondant technique pour les lois de comportement gaine du projet CRAYON \\
  {\tiny 2013-????} & Responsable du développement de MFront \\
  {\tiny 2013-2015} & DEC/SESC/LSC: Développement GERMINAL (éléments absorbants, composants RAMSES et thermohydraulique) \\
  {\tiny 2009-2014} & DEC/SESC/LSC: Responsable de développement de Licos : applications pour la conception des éléments combustibles et absorbants innovants, des irradiations non standards et à l’interprétation des irradations non standards \\
  {\tiny 2010-2008} & DEC/SESC/LSC: Responsable de dévéloppement du code de calcul CELAENO pour les éléménts combustible de la filière RNR-Gaz et le design des irradiations expérimentales associées \\
\end{tabular}

\section{Formation}

\subsection{Formation initiale}

\cventry{2005}{Docteur en mécaniques des solides}{École centrale de
  Lyon}{France}{Étude de l'impact de la fissuration des combustibles
  nucléaires oxyde sur le comportement normal et incidentel des
  crayons combustible}{}

\cventry{2002}{DEA en Sciences des Matériaux}{Université Henri
  Poincaré (Nancy I)}{France}{}{}

\cventry{2002}{Ingénieur}{École des Mines de Nancy}{France}{Spécialité
  en science et ingénierie des matériaux}{}

\subsection{Formation professionnelle}
	
\begin{tabular}[htbp]{p{0.1\linewidth}p{0.9\linewidth}}
  2016 & Contrôler et améliorer la qualité numérique d'un code de calcul industriel \\
  2015 & Animez et optimisez vos réunions\\
  2014 & Formation Salome (GEOM/SMESH - PARAVIS - MEDCOUPLING) \\
  2012 & Salome CAO maillage ET Paravis \\
  2011 & ASTRID : R\&D et projet, cohérence et complétude \\
  2010 & Les réacteurs à neutrons rapides refroidis au sodium - module général \\
  2010 & Gérer ses émotions et renforcer sa confiance en soi, lors d'une prise de parole en public \\
  2009 & SPIRALE - FNB niv.3 Formation Nucléaire de Base \\
  2008 & SPIRALE - FNB niv.2 L'Electronucléaire et les Installations Nucléaires  \\
  2008 & SPIRALE - FNB niv. 1 Introduction à l'Energie Nucléaire \\
  2006 & Conception d'applications scientifiques hautes performances \\
  2006 & Programmation C++ \\
  2006 & Théorie appliquée aux réacteurs PN \\
  2004 & École d'été MEALOR (Mécanique de l'endommagement et approche locale de la rupture)\\
\end{tabular}

\section{Activités d'enseignement, Formations}

\subsection{Enseignement}

\begin{tabular}[htbp]{p{0.1\linewidth}p{0.9\linewidth}}
  2016      & Materials science for Nuclear Energy (MaNuEn)/European Master in Nuclear Energy (EMINE) \\
            & Non linear mechanical phenomena in fuel elements simulation (12h) \\
  2016      & Polytech’Marseille, option \og{}~Structures et Ouvrage\fg{}, 3ème année cursus ingénieur\\
            & Mécanique non linéaire et \texttt{Cast3M} : 13h de cours/TD. \\
  {\tiny 2010-2015} & Materials science for Nuclear Energy (MaNuEn)/European Master in Nuclear Energy (EMINE) \\
            & Basic modelling of a fuel rod and application to fuel element design (8h) \\
\end{tabular}

\subsection{Formations}

\begin{tabular}[htbp]{p{0.1\linewidth}p{0.9\linewidth}}
  2016      & Formation MFront (DMN) \\
  2016      & Formation MFront (DM2S) \\
  2016      & Formation MFront (DEC) \\
  2016      & Formation C++ (DEC/SESC) \\
\end{tabular}

\section{Encadrement}

\begin{tabular}[htbp]{p{0.1\linewidth}p{0.9\linewidth}}
  2017 & Antonin Aguilar, stage Université de technologie de Troyes (6mois) : {\em Identification robuste des lois de comportement viscoplastique du combustible nucléaire oxyde UO2 des réacteurs à eau pressurisée} \\
  2016 & Rémy Cnocquart, stage INSA-Lyon (6mois) : {\em modélisation non locale de la propagation de fissure dans le combustible nucléaire : applications à la fragmentation du combustible} \\
  2016 & Tom Maurel, stage INSA-Lyon (6mois) : {\em modélisation du collage des pastilles de combustible nucléaire fortement irradiée au tube de gainage} \\
  2015 & Sébastien Melin, stage Ecole Polytechnique (6 mois) : {\em modélisation de la rupture d'une gaine de crayon combustible au cours d'un accident d'injection de réactivité} \\
  2013 & Matthieu Occelli, stage INSA-Lyon (6mois) : {\em dimensionnement des éléments absorbants du réacteur ASTRID en séisme : modélisation poutre et analyse locale par raccord poutre-massif} \\
  2008 & Thomas Roncaglia, stage DUT (3 mois) : {\em parallélisation des modèles de la plate-forme PLEIADES} \\
  2008 & Sébastien Muller, stage Ecole Polytechnique (6 mois) : {\em influence des fissures radiales dans la pastille combustible sur le comportement mécanique du crayon combustible, apports d’une modélisation 2D (r, \(\theta\))}  \\
  2007 & Matthieu Turban, stage Ecole Polytechnique (6 mois) : {\em étude du comportement thermoélastique d’un fragment de pastille combustible sous irradiation : résolution par la méthode du prolongement analytique} \\
\end{tabular}

\section{Site internet}

\begin{center}
  \includegraphics[width=0.5\linewidth]{img/mfront-website.png}
\end{center}

Le site internet de \texttt{MFront}~:
\begin{center}
  \url{http://tfel.sourceforge.net}  
\end{center}

\section{Animation scientifique}

\subsection{Club utilisateurs MFront}

\begin{tabular}[htbp]{p{0.1\linewidth}p{0.9\linewidth}}
  2017 &Troisième journée des utilisateurs \texttt{MFront}, DIGITEO (\(\approx{}35\) participants). \\
  & Les présentations sont accessibles sur le dépôt~: \\
  & \url{https://github.com/thelfer/tfel-doc/tree/master/MFrontUserDays/ThirdUserDay}\\
  2016 &Seconde journée des utilisateurs \texttt{MFront}, EDF Lab Saclay (\(\approx{}60\) participants). \\
  & Les présentations sont accessibles sur le dépôt~: \\
  & \url{https://github.com/thelfer/tfel-doc/tree/master/MFrontUserDays/SecondUserDay}\\
  2015 &Première journée des utilisateurs \texttt{MFront}, CEA Cadarache (\(35\) participants). \\
  & Les présentations sont accessibles sur le dépôt~: \\
  & \url{https://github.com/thelfer/tfel-doc/tree/master/MFrontUserDays/FirstUserDay}\\
\end{tabular}

\section{Faits marquants}

\subsection{Faits marquants DEN ayant fait l'objet d'un FLASH DEN}

\begin{tabular}[htbp]{p{0.1\linewidth}p{0.9\linewidth}}
  2017 &	Livraison au format MFRONT de la loi EDGAR M5 v3 décrivant le comportement mécanique des gaines M5® en conditions d’accident de perte de réfrigérant primaire \\
         
  2014 &	La plateforme PLEIADES diffuse en Open-source le composant MFRONT \\
\end{tabular}

\subsection{Faits marquants DEN}

\begin{tabular}[htbp]{p{0.1\linewidth}p{0.9\linewidth}}
  2013 & Application conception PLEIADES (LICOS)~: \\ 
       & première formation « Utilisateurs » à AREVA \\
  2012 & Livraison du Modèle LICOS-DIAMINO du DEC à DRSN/SIREN \\
\end{tabular}

\subsection{Faits marquants DEC}

\begin{tabular}[htbp]{p{0.1\linewidth}p{0.9\linewidth}}
  2018 & Modélisation des expériences menées au sur le combustible REP en conditions de transport et d'entreposage \\
  2017 & Livraison des lois EDGAR V3 au format MFront par le DMN/SRMA \\
  2016 & Seconde journée des utilisateurs MFront \\
  2015 & Livraison de l’application LICOS 1.2 \\
  2013 & Séminaire en l’honneur des 10 ans de PLEIADES \\
\end{tabular}

\bibliographystylearticles{unsrturl}
\bibliographystyleproceedings{unsrturl}
\bibliographystylenotes{unsrturl}
\bibliographystyletalks{unsrturl}
\bibliographystylemfront{unsrturl}

\nocitearticles{guenot-delahaie_simulation_2018}
\nocitearticles{michel_new_2017}
\nocitearticles{bary_analytical_2017}
\nocitearticles{bejaoui17:_therm_diamin_licos}
\nocitearticles{michel16}
\nocitearticles{bary_thermoviscoelastic_2016}
\nocitearticles{sercombe_2d_2016}
\nocitearticles{helfer_licos_2015}
\nocitearticles{helfer_extension_2015}
\nocitearticles{helfer_introducing_2015}
\nocitearticles{ramiere_iterative_2015-1}
\nocitearticles{salvo_experimental_2015}
\nocitearticles{salvo_experimental_2015-1}
\nocitearticles{sercombe_stress_2013}

\nociteproceedings{*}
\nocitenotes{*}
\nocitetalks{*}
\nocitemfront{*}

\bibliographyarticles{bibliography}
\bibliographyproceedings{proceedings}
\bibliographynotes{notes}
\bibliographytalks{talks,proceedings}
\bibliographymfront{mfront}

\end{document}

% Publications
% Thomas Helfer
% 16 septembre 2016

% # Articles de revue

% - @bary_thermoviscoelastic_2016
% - @michel_3d_2016
% - @sercombe_2d_2016
% - @helfer_implantation_2015
% - @helfer_licos_2015
% - @helfer_extension_2015
% - @helfer_introducing_2015
% - @ramiere_iterative_2015
% - @salvo_experimental_2015
% - @salvo_experimental_2015-1
% - @sercombe_stress_2013

% # Congrès

% - @castelier_using_2016
% - @helfer_recent_2015
% - @bary_numerical_2015
% - @blanc_characterization_2015
% - @petry_advanced_2015
% - @sercombe_2d_2015
% - @lainet_recent_2013
% - @bejaoui_thermomechanical_2013
% - @helfer_fuel_2009
% - @helfer_two-dimensional_2006
% - @plancq_pleiades_2004

% # Thèse

% - @helfer_etude_2006

% # Bibliographie

%%% Local Variables:
%%% mode: latex
%%% TeX-master: t
%%% End:
